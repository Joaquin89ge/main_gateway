%% Generated by Sphinx.
\def\sphinxdocclass{report}
\documentclass[a4paper,11pt,spanish]{sphinxmanual}
\ifdefined\pdfpxdimen
   \let\sphinxpxdimen\pdfpxdimen\else\newdimen\sphinxpxdimen
\fi \sphinxpxdimen=.75bp\relax
\ifdefined\pdfimageresolution
    \pdfimageresolution= \numexpr \dimexpr1in\relax/\sphinxpxdimen\relax
\fi
%% let collapsible pdf bookmarks panel have high depth per default
\PassOptionsToPackage{bookmarksdepth=5}{hyperref}
%% turn off hyperref patch of \index as sphinx.xdy xindy module takes care of
%% suitable \hyperpage mark-up, working around hyperref-xindy incompatibility
\PassOptionsToPackage{hyperindex=false}{hyperref}
%% memoir class requires extra handling
\makeatletter\@ifclassloaded{memoir}
{\ifdefined\memhyperindexfalse\memhyperindexfalse\fi}{}\makeatother

\PassOptionsToPackage{booktabs}{sphinx}
\PassOptionsToPackage{colorrows}{sphinx}

\PassOptionsToPackage{warn}{textcomp}

\catcode`^^^^00a0\active\protected\def^^^^00a0{\leavevmode\nobreak\ }
\usepackage{cmap}
\usepackage{fontspec}
\defaultfontfeatures[\rmfamily,\sffamily,\ttfamily]{}
\usepackage{amsmath,amssymb,amstext}
\usepackage{polyglossia}
\setmainlanguage{spanish}



\setmainfont{FreeSerif}[
  Extension      = .otf,
  UprightFont    = *,
  ItalicFont     = *Italic,
  BoldFont       = *Bold,
  BoldItalicFont = *BoldItalic
]
\setsansfont{FreeSans}[
  Extension      = .otf,
  UprightFont    = *,
  ItalicFont     = *Oblique,
  BoldFont       = *Bold,
  BoldItalicFont = *BoldOblique,
]
\setmonofont{FreeMono}[Scale=0.9,
  Extension      = .otf,
  UprightFont    = *,
  ItalicFont     = *Oblique,
  BoldFont       = *Bold,
  BoldItalicFont = *BoldOblique,
]



\usepackage[Sonny]{fncychap}
\ChNameVar{\Large\normalfont\sffamily}
\ChTitleVar{\Large\normalfont\sffamily}
\usepackage{sphinx}

\fvset{fontsize=auto}
\usepackage{geometry}


% Include hyperref last.
\usepackage{hyperref}
% Fix anchor placement for figures with captions.
\usepackage{hypcap}% it must be loaded after hyperref.
% Set up styles of URL: it should be placed after hyperref.
\urlstyle{same}


\usepackage{sphinxmessages}



        \usepackage{charter}
        \usepackage[defaultsans]{lato}
        \usepackage{inconsolata}
    

\title{Documentación del Sistema Gateway Agrícola}
\date{20 de julio de 2025}
\release{1.0.0}
\author{Equipo de Desarrollo Agrícola}
\newcommand{\sphinxlogo}{\vbox{}}
\renewcommand{\releasename}{Versión}
\makeindex
\begin{document}

\pagestyle{empty}
\sphinxmaketitle
\pagestyle{plain}
\sphinxtableofcontents
\pagestyle{normal}
\phantomsection\label{\detokenize{index::doc}}



\chapter{📚 Índice de Documentación}
\label{\detokenize{index:indice-de-documentacion}}

\section{📋 Descripción General}
\label{\detokenize{index:descripcion-general}}\begin{itemize}
\item {} 
\sphinxAtStartPar
\sphinxstylestrong{\DUrole{std}{\DUrole{std-doc}{README.md}}} \sphinxhyphen{} Visión general del sistema Gateway agrícola con red mesh LoRa

\end{itemize}


\section{🔧 Componentes Principales}
\label{\detokenize{index:componentes-principales}}

\subsection{Archivos de Sistema}
\label{\detokenize{index:archivos-de-sistema}}\begin{itemize}
\item {} 
\sphinxAtStartPar
\sphinxstylestrong{\DUrole{std}{\DUrole{std-doc}{main\_gateway.md}}} \sphinxhyphen{} Punto de entrada del sistema Gateway

\item {} 
\sphinxAtStartPar
\sphinxstylestrong{\DUrole{std}{\DUrole{std-doc}{config.md}}} \sphinxhyphen{} Configuración central del sistema

\end{itemize}


\subsection{Clases del Sistema}
\label{\detokenize{index:clases-del-sistema}}\begin{itemize}
\item {} 
\sphinxAtStartPar
\sphinxstylestrong{\DUrole{std}{\DUrole{std-doc}{node\_identity.md}}} \sphinxhyphen{} Gestión de identidad única del Gateway

\item {} 
\sphinxAtStartPar
\sphinxstylestrong{\DUrole{std}{\DUrole{std-doc}{radio\_manager.md}}} \sphinxhyphen{} Gestión de comunicación LoRa mesh

\item {} 
\sphinxAtStartPar
\sphinxstylestrong{\DUrole{std}{\DUrole{std-doc}{rtc\_manager.md}}} \sphinxhyphen{} Gestión de tiempo real con DS1302

\item {} 
\sphinxAtStartPar
\sphinxstylestrong{\DUrole{std}{\DUrole{std-doc}{app\_logic.md}}} \sphinxhyphen{} Lógica de aplicación del Gateway

\end{itemize}


\chapter{🏗️ Arquitectura del Sistema}
\label{\detokenize{index:arquitectura-del-sistema}}

\section{Componentes Hardware}
\label{\detokenize{index:componentes-hardware}}\begin{itemize}
\item {} 
\sphinxAtStartPar
\sphinxstylestrong{ESP8266:} Microcontrolador principal

\item {} 
\sphinxAtStartPar
\sphinxstylestrong{SX1278 LoRa:} Módulo de comunicación de radio

\item {} 
\sphinxAtStartPar
\sphinxstylestrong{DS1302 RTC:} Reloj de tiempo real

\item {} 
\sphinxAtStartPar
\sphinxstylestrong{LittleFS:} Sistema de archivos para persistencia

\end{itemize}


\section{Componentes Software}
\label{\detokenize{index:componentes-software}}\begin{itemize}
\item {} 
\sphinxAtStartPar
\sphinxstylestrong{NodeIdentity:} Gestión de identidad única

\item {} 
\sphinxAtStartPar
\sphinxstylestrong{RadioManager:} Comunicación LoRa mesh

\item {} 
\sphinxAtStartPar
\sphinxstylestrong{RtcManager:} Gestión de tiempo real

\item {} 
\sphinxAtStartPar
\sphinxstylestrong{AppLogic:} Lógica de aplicación central

\end{itemize}


\chapter{📊 Especificaciones Técnicas}
\label{\detokenize{index:especificaciones-tecnicas}}

\section{Hardware}
\label{\detokenize{index:hardware}}\begin{itemize}
\item {} 
\sphinxAtStartPar
\sphinxstylestrong{Microcontrolador:} ESP8266 80MHz

\item {} 
\sphinxAtStartPar
\sphinxstylestrong{Memoria RAM:} 80KB

\item {} 
\sphinxAtStartPar
\sphinxstylestrong{Memoria Flash:} 4MB

\item {} 
\sphinxAtStartPar
\sphinxstylestrong{Comunicación:} LoRa 433MHz

\item {} 
\sphinxAtStartPar
\sphinxstylestrong{RTC:} DS1302 con interfaz 3\sphinxhyphen{}wire

\end{itemize}


\section{Software}
\label{\detokenize{index:software}}\begin{itemize}
\item {} 
\sphinxAtStartPar
\sphinxstylestrong{Arquitectura:} Modular con principios SOLID

\item {} 
\sphinxAtStartPar
\sphinxstylestrong{Comunicación:} Protocolo LoRa mesh personalizado

\item {} 
\sphinxAtStartPar
\sphinxstylestrong{Persistencia:} LittleFS para configuración

\item {} 
\sphinxAtStartPar
\sphinxstylestrong{Escalabilidad:} Hasta 250 nodos

\end{itemize}


\chapter{🚀 Funcionalidades Principales}
\label{\detokenize{index:funcionalidades-principales}}

\section{1. Gestión de Identidad}
\label{\detokenize{index:gestion-de-identidad}}\begin{itemize}
\item {} 
\sphinxAtStartPar
Generación de ID único basado en MAC address

\item {} 
\sphinxAtStartPar
Protección contra colisiones con lista negra

\item {} 
\sphinxAtStartPar
Persistencia en LittleFS

\item {} 
\sphinxAtStartPar
Gestión de gateway asociado

\end{itemize}


\section{2. Comunicación Mesh}
\label{\detokenize{index:comunicacion-mesh}}\begin{itemize}
\item {} 
\sphinxAtStartPar
Protocolo LoRa personalizado

\item {} 
\sphinxAtStartPar
Enrutamiento automático con RHMesh

\item {} 
\sphinxAtStartPar
Retransmisión automática

\item {} 
\sphinxAtStartPar
Gestión de congestión de red

\end{itemize}


\section{3. Gestión de Tiempo}
\label{\detokenize{index:gestion-de-tiempo}}\begin{itemize}
\item {} 
\sphinxAtStartPar
Sincronización con RTC DS1302

\item {} 
\sphinxAtStartPar
Comparación de horarios para eventos

\item {} 
\sphinxAtStartPar
Configuración automática de fecha/hora

\item {} 
\sphinxAtStartPar
Validación de integridad temporal

\end{itemize}


\section{4. Lógica de Aplicación}
\label{\detokenize{index:logica-de-aplicacion}}\begin{itemize}
\item {} 
\sphinxAtStartPar
Coordinación centralizada de nodos

\item {} 
\sphinxAtStartPar
Almacenamiento de muestras por nodo

\item {} 
\sphinxAtStartPar
Programación temporal de eventos

\item {} 
\sphinxAtStartPar
Monitoreo de estado de nodos

\end{itemize}


\chapter{📈 Métricas de Rendimiento}
\label{\detokenize{index:metricas-de-rendimiento}}

\section{Comunicación}
\label{\detokenize{index:comunicacion}}\begin{itemize}
\item {} 
\sphinxAtStartPar
\sphinxstylestrong{Alcance:} Hasta 10km en condiciones óptimas

\item {} 
\sphinxAtStartPar
\sphinxstylestrong{Throughput:} Hasta 37.5 kbps

\item {} 
\sphinxAtStartPar
\sphinxstylestrong{Latencia:} <2 segundos

\item {} 
\sphinxAtStartPar
\sphinxstylestrong{Nodos Máximos:} 250 nodos

\end{itemize}


\section{Recursos}
\label{\detokenize{index:recursos}}\begin{itemize}
\item {} 
\sphinxAtStartPar
\sphinxstylestrong{RAM:} \textasciitilde{}44\% (36KB de 80KB)

\item {} 
\sphinxAtStartPar
\sphinxstylestrong{Flash:} \textasciitilde{}32\% (340KB de 1MB)

\item {} 
\sphinxAtStartPar
\sphinxstylestrong{Energía:} Optimizado para eficiencia

\item {} 
\sphinxAtStartPar
\sphinxstylestrong{CPU:} Bajo impacto en operación

\end{itemize}


\chapter{🔍 Características Técnicas}
\label{\detokenize{index:caracteristicas-tecnicas}}

\section{1. Arquitectura Modular}
\label{\detokenize{index:arquitectura-modular}}\begin{itemize}
\item {} 
\sphinxAtStartPar
\sphinxstylestrong{Single Responsibility:} Cada clase tiene una responsabilidad específica

\item {} 
\sphinxAtStartPar
\sphinxstylestrong{Open/Closed:} Extensible sin modificar código existente

\item {} 
\sphinxAtStartPar
\sphinxstylestrong{Liskov Substitution:} Interfaces consistentes

\item {} 
\sphinxAtStartPar
\sphinxstylestrong{Interface Segregation:} Interfaces específicas por funcionalidad

\item {} 
\sphinxAtStartPar
\sphinxstylestrong{Dependency Inversion:} Dependencias a través de abstracciones

\end{itemize}


\section{2. Gestión de Errores}
\label{\detokenize{index:gestion-de-errores}}\begin{itemize}
\item {} 
\sphinxAtStartPar
\sphinxstylestrong{Validación de rangos} para todos los parámetros

\item {} 
\sphinxAtStartPar
\sphinxstylestrong{Detección de fallos} de hardware

\item {} 
\sphinxAtStartPar
\sphinxstylestrong{Recuperación automática} de errores

\item {} 
\sphinxAtStartPar
\sphinxstylestrong{Logging detallado} para debugging

\end{itemize}


\section{3. Optimización de Energía}
\label{\detokenize{index:optimizacion-de-energia}}\begin{itemize}
\item {} 
\sphinxAtStartPar
\sphinxstylestrong{WiFi deshabilitado} para reducir interferencias

\item {} 
\sphinxAtStartPar
\sphinxstylestrong{Transmisión selectiva} de datos críticos

\item {} 
\sphinxAtStartPar
\sphinxstylestrong{Gestión de buffer} eficiente

\item {} 
\sphinxAtStartPar
\sphinxstylestrong{Monitoreo continuo} de estado de red

\end{itemize}


\chapter{🎯 Casos de Uso}
\label{\detokenize{index:casos-de-uso}}

\section{1. Monitoreo de Red Agrícola}
\label{\detokenize{index:monitoreo-de-red-agricola}}\begin{itemize}
\item {} 
\sphinxAtStartPar
\sphinxstylestrong{Coordinación} de múltiples nodos sensores

\item {} 
\sphinxAtStartPar
\sphinxstylestrong{Consolidación} de datos ambientales

\item {} 
\sphinxAtStartPar
\sphinxstylestrong{Gestión} de muestras de suelo

\item {} 
\sphinxAtStartPar
\sphinxstylestrong{Optimización} de comunicación mesh

\end{itemize}


\section{2. Gestión de Invernaderos}
\label{\detokenize{index:gestion-de-invernaderos}}\begin{itemize}
\item {} 
\sphinxAtStartPar
\sphinxstylestrong{Control centralizado} de sensores

\item {} 
\sphinxAtStartPar
\sphinxstylestrong{Monitoreo} de condiciones críticas

\item {} 
\sphinxAtStartPar
\sphinxstylestrong{Alertas} de problemas de red

\item {} 
\sphinxAtStartPar
\sphinxstylestrong{Registro} histórico de datos

\end{itemize}


\section{3. Agricultura de Precisión}
\label{\detokenize{index:agricultura-de-precision}}\begin{itemize}
\item {} 
\sphinxAtStartPar
\sphinxstylestrong{Mapeo} de variabilidad del suelo

\item {} 
\sphinxAtStartPar
\sphinxstylestrong{Coordinación} de sensores distribuidos

\item {} 
\sphinxAtStartPar
\sphinxstylestrong{Optimización} de recursos de red

\item {} 
\sphinxAtStartPar
\sphinxstylestrong{Análisis} de tendencias centralizadas

\end{itemize}


\chapter{🔮 Roadmap Técnico}
\label{\detokenize{index:roadmap-tecnico}}

\section{Fase 1 (Completada)}
\label{\detokenize{index:fase-1-completada}}\begin{itemize}
\item {} 
\sphinxAtStartPar
✅ Arquitectura modular básica

\item {} 
\sphinxAtStartPar
✅ Sistema de comunicación LoRa mesh

\item {} 
\sphinxAtStartPar
✅ Gestión de identidad única

\item {} 
\sphinxAtStartPar
✅ RtcManager con DS1302

\end{itemize}


\section{Fase 2 (En Desarrollo)}
\label{\detokenize{index:fase-2-en-desarrollo}}\begin{itemize}
\item {} 
\sphinxAtStartPar
🔄 Documentación completa con Sphinx

\item {} 
\sphinxAtStartPar
🔄 Optimización de consumo energético

\item {} 
\sphinxAtStartPar
🔄 Interfaz web para configuración

\item {} 
\sphinxAtStartPar
🔄 Sistema de alertas avanzado

\end{itemize}


\section{Fase 3 (Planificada)}
\label{\detokenize{index:fase-3-planificada}}\begin{itemize}
\item {} 
\sphinxAtStartPar
📋 Integración con sistemas de riego

\item {} 
\sphinxAtStartPar
📋 Interfaz móvil para monitoreo

\item {} 
\sphinxAtStartPar
📋 Análisis avanzado de datos

\item {} 
\sphinxAtStartPar
📋 Integración con IA para predicciones

\end{itemize}


\chapter{📋 Guías de Uso}
\label{\detokenize{index:guias-de-uso}}

\section{Configuración Inicial}
\label{\detokenize{index:configuracion-inicial}}\begin{enumerate}
\sphinxsetlistlabels{\arabic}{enumi}{enumii}{}{.}%
\item {} 
\sphinxAtStartPar
\sphinxstylestrong{Hardware:} Conectar componentes según diagrama

\item {} 
\sphinxAtStartPar
\sphinxstylestrong{Software:} Compilar y subir firmware

\item {} 
\sphinxAtStartPar
\sphinxstylestrong{Configuración:} Ajustar parámetros en config.h

\item {} 
\sphinxAtStartPar
\sphinxstylestrong{Pruebas:} Verificar comunicación y funcionamiento

\end{enumerate}


\section{Operación Diaria}
\label{\detokenize{index:operacion-diaria}}\begin{enumerate}
\sphinxsetlistlabels{\arabic}{enumi}{enumii}{}{.}%
\item {} 
\sphinxAtStartPar
\sphinxstylestrong{Monitoreo:} Verificar estado de nodos

\item {} 
\sphinxAtStartPar
\sphinxstylestrong{Datos:} Revisar muestras recibidas

\item {} 
\sphinxAtStartPar
\sphinxstylestrong{Mantenimiento:} Limpiar buffers y logs

\item {} 
\sphinxAtStartPar
\sphinxstylestrong{Optimización:} Ajustar parámetros según necesidades

\end{enumerate}


\section{Troubleshooting}
\label{\detokenize{index:troubleshooting}}\begin{enumerate}
\sphinxsetlistlabels{\arabic}{enumi}{enumii}{}{.}%
\item {} 
\sphinxAtStartPar
\sphinxstylestrong{Comunicación:} Verificar pines y antenas

\item {} 
\sphinxAtStartPar
\sphinxstylestrong{RTC:} Validar conexiones y batería

\item {} 
\sphinxAtStartPar
\sphinxstylestrong{Memoria:} Monitorear uso de RAM

\item {} 
\sphinxAtStartPar
\sphinxstylestrong{Energía:} Verificar consumo y estabilidad

\end{enumerate}


\chapter{🔗 Enlaces Relacionados}
\label{\detokenize{index:enlaces-relacionados}}

\section{Documentación Externa}
\label{\detokenize{index:documentacion-externa}}\begin{itemize}
\item {} 
\sphinxAtStartPar
\sphinxstylestrong{RadioHead Library:} Documentación de RHMesh

\item {} 
\sphinxAtStartPar
\sphinxstylestrong{ESP8266 Reference:} Guía del microcontrolador

\item {} 
\sphinxAtStartPar
\sphinxstylestrong{DS1302 Datasheet:} Especificaciones del RTC

\item {} 
\sphinxAtStartPar
\sphinxstylestrong{LoRa Technology:} Información sobre LoRa

\end{itemize}


\section{Recursos de Desarrollo}
\label{\detokenize{index:recursos-de-desarrollo}}\begin{itemize}
\item {} 
\sphinxAtStartPar
\sphinxstylestrong{PlatformIO:} Entorno de desarrollo

\item {} 
\sphinxAtStartPar
\sphinxstylestrong{Arduino IDE:} Alternativa de desarrollo

\item {} 
\sphinxAtStartPar
\sphinxstylestrong{GitHub Repository:} Código fuente completo

\item {} 
\sphinxAtStartPar
\sphinxstylestrong{Issue Tracker:} Reporte de problemas

\end{itemize}


\chapter{📞 Soporte}
\label{\detokenize{index:soporte}}

\section{Contacto}
\label{\detokenize{index:contacto}}\begin{itemize}
\item {} 
\sphinxAtStartPar
\sphinxstylestrong{Desarrollador:} Equipo de desarrollo agrícola

\item {} 
\sphinxAtStartPar
\sphinxstylestrong{Email:} soporte@agro\sphinxhyphen{}iot.com

\item {} 
\sphinxAtStartPar
\sphinxstylestrong{GitHub:} https://github.com/agro\sphinxhyphen{}iot/gateway

\item {} 
\sphinxAtStartPar
\sphinxstylestrong{Documentación:} https://docs.agro\sphinxhyphen{}iot.com

\end{itemize}


\section{Comunidad}
\label{\detokenize{index:comunidad}}\begin{itemize}
\item {} 
\sphinxAtStartPar
\sphinxstylestrong{Foro:} Comunidad de usuarios

\item {} 
\sphinxAtStartPar
\sphinxstylestrong{Wiki:} Documentación colaborativa

\item {} 
\sphinxAtStartPar
\sphinxstylestrong{Tutoriales:} Guías paso a paso

\item {} 
\sphinxAtStartPar
\sphinxstylestrong{Ejemplos:} Código de ejemplo

\end{itemize}


\bigskip\hrule\bigskip


\sphinxAtStartPar
\sphinxstylestrong{Nota:} Esta documentación está diseñada para ser expandible y mejorable. Cada componente tiene su propia documentación detallada que puede ser actualizada independientemente según las necesidades del proyecto.



\renewcommand{\indexname}{Índice}
\printindex
\end{document}